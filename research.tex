\documentclass{article}
\usepackage{float}
\usepackage{graphicx}
%\usepackage{lipsum}
%\usepackage[margin=1in,includefoot]{geometry}

\begin{document}
\begin{titlepage}
	\begin{center}
	\line(1,0){300}\\
	[0.25in]
	\huge{\bfseries The Effect of Drug Abuse on Academic Performance in Secondary Schools in Uganda}\\
	[2mm]
	\line(1,0){200}\\
		\begin{figure}[H]
		\centering
		\includegraphics[height=3in]{/Users/Dintaine/Desktop/latex/figures/drug.jpg}
		
		\end{figure}
	\end{center}
\begin{flushright}
\textsc{\large Ainemukama Dinton Harold}
 
 COMPUTER SCIENCE YEAR $2$\\
$216007270$\\
$16/U/3020/PS$\\
Feb 4, 2018\\
\end{flushright}
	
\end{titlepage}
\tableofcontents
\thispagestyle{empty}
\cleardoublepage

\setcounter{page}{1}
\section{Background of the research}
Drug abuse among teens is the great problem that has speeded all over the world. In Jamaica, the use of drugs abuse by teenagers has more increased over decades in studying the drug usage patterns of Jamaica teens discovered that while usage was not dependent on sex.In $1989$, $78$ percent of teen males and percent of teen female were using one of the four drugs (alcohol,marijuana,cocaine and tobacco) between $1994$ and $1995$;it indicates that $60$ percent of the teens have tried one or more drugs including marijuana while $1.3$ percent has used cocaine.In Uganda also the problem has speeded at large percents whereby the students from secondary schools have been noted for abusing drugs.The adolescents who abuse drugs  often act out do poorly academically and drop out of school.They are at a risk of unplanned pregnancies, violence and infectious diseases.
\section{Problem Statement}
Drug abuse among teens and school students seem to have detrimental effect on the academic performance. Youth are more susceptible to the short and long term cognitive effect of drug abuse while the social and emotional repercussions further increase risk factor for problem in school. It is important for the parents and students to learn the risk of the drug use and take action as soon as concerns are raised to avoid serious problems in school. This problem has become an  issue in secondary schools in Uganda.Hence the study wants to investigate its effect on academic performance.
\section{Main Objectives}
To find out the effects of drug abuse on academic performance in Secondary schools in Uganda.
\section{Specific Objectives}
To identify the causes of teens to engage in drug abuse.\\
To identify the impacts of drug abuse on academic performance.\\
To suggest the possible solutions towards drug abuse.
\section{Introduction}\label{sec:intro}
Drug abuse, also called substance abuse or chemical abuse is a disorder that is characterized by a destructive pattern of using substance that leads to significant problems or distress. Teens are increasingly engaging in prescription drug abuse. It leads to significant problems that use of substance can cause for the sufferer, either socially or in terms of their work or school performance. 
\section{Types of Drugs commonly Abused}
Any substance whose ingestion can result in high feeling can be abused. The following are many drugs and types of drugs that are commonly abused or result in dependence:
Alcohol though legal yet is dangerous. Anabolic steroids, abused by students who play games that involve energy for example rugby. This group of drugs can lead to terrible psychological  effects like aggression and devastating long term physical effects like infertility and organ failure. Caffeine  is consumed by many coffee, tea and soda drinkers, when consumed in excess this substance can produce palpitations ( rapid and irregular heartbeat), insomnia(sleeplessness), tremors(involuntary vibration of body), anxiety(nervousness, attacks of panic). Cocaine tends to stimulate the nervous system. It is smoked and as well as injected. This a very expensive drug and is normally abused by foreigner students who smuggle it into the country. Nicotine  is the addictive substance found in cigarettes. Its actually one of the most habit-forming substances that exists.And so many other drugs like kuba,mairungi and so many others.
\section{ Effects of Drug Abuse to Academic Performance}
Like the majority of other mental-health problems, drug abuse and addiction have no single cause. However, there are number of biological, psychological and social factors called risk factors that can increase a person’s likelihood of developing a chemical-abuse or chemical dependency disorder.While each drug produces different physical effects, all abused substances share one thing in common: repeated use can alter the way the brain looks and functions. Common effects of drug use on the brain that impact academic learning includes difficulty concetrating, inability to process information and problems with working memory.In $2012$ in my senior three at Uganda Martyrs SS Namugongo, we had a drug awareness talk and $50$percent of my classmates had used drugs atleast once in their lifetime especially alcohol.The frequency to which substance abuse occur within some families seems to be higher than could be explained by an addictive environment of the family. Some professionals recognize a genetic aspect to the risk of drug addiction especially the case of alcohol. One of the most harmful risks is that of engaging in risky sexual activities. The use of drugs is related to the occurrence of unsafe sexual behavior that places adolescent at risk for pregnancy of contracting sexually transmitted diseases such as HIV/AIDS.During the conducting of the research,I used Quantitative and Qualitative methods,this is because I was aiming at getting both statistical and descriptive data towards the problem.Due to this, I was in position of accumulating more data as inteded to solve the problem.



\end{document}